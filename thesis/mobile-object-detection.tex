%!TeX spellcheck=en_GB
\documentclass[11pt,
               a4paper,
               bibtotoc,
               idxtotoc,
               headsepline,
               footsepline,
               footexclude,
               BCOR12mm,
               DIV13,
               openany,   % using this removes blank pages around part / chapter starts.
%               oneside    % include this if you have to print only one page per sheet of paper.
               ]
               {scrbook}

%%% SETTINGS

% no word wrapping
%\righthyphenmin=62
%\lefthyphenmin=62
% fewer hyphens
\usepackage{microtype}

% german symbols
\usepackage[utf8]{inputenc}

% strikethrough by \sout
\usepackage[normalem]{ulem}

% insert graphics
\usepackage{graphicx}
% more flexible figures e.g. graphics with captions beside them
\usepackage{floatrow}
% more flexible captions.
% Use \captionsetup{options} to configure,
% use it in an environment for local setup
\usepackage{caption}
% subfigures (see template):
\usepackage{subcaption}

% more control of enumerations and itemizations
\usepackage{enumitem}
% less space between items
\setlist[itemize]{itemsep=0cm}
\setlist[enumerate]{itemsep=0cm}
% more customizeable tables (e.g. multiple lines per cell)
\usepackage{tabularx}
% fix for vertical centering
\usepackage{ragged2e}
\renewcommand\tabularxcolumn[1]{>{\Centering}m{#1}}
% column types with multiple lines and formatting
\usepackage{array}
\newcolumntype{C}{>{\centering\arraybackslash}X}
\newcolumntype{R}{>{\raggedleft\arraybackslash}X}
\newcolumntype{L}{>{\raggedright\arraybackslash}X}
% merge multiple rows \multirow{2}{*}{bla} & \\ &
\usepackage{multirow}
% activate for tables with page breaking
%\usepackage{ltablex}
% fix for table movement and itemizations
%\keepXColumns

% fix for dynamics spaces after custom commands
\usepackage{xspace}

% tabbing: use with \tab
\usepackage{tabto}
\TabPositions{4cm}

%% fancy math
% propper matrices, underbrace text
%\usepackage{amsmath}
\usepackage{mathtools}
% special symbols e.g. squares
\usepackage{amssymb}

% code coloring
\usepackage[dvipsnames]{xcolor} % must be declared before pgfplots, tikz or pstricks package, otherwise it will not work.

%% plotting
\usepackage{pgfplots}
\usepgfplotslibrary{fillbetween}

%%Settings for code
% code placement right there
\usepackage{float}
% code listing
\usepackage{listings}

% flexible multi column style
\usepackage{multicol}

% graphs
\usepackage{tikz}
\usetikzlibrary{shapes.geometric, arrows}
% define some elements
\tikzstyle{box} = [rectangle, rounded corners, minimum width=3cm, minimum height=1cm,text centered, draw=black, fill=black!5]
\tikzstyle{arrow} = [thick,->,>=stealth]
\usepackage{varwidth}

% Some code highlighting styles you can use with lstlistings
% C++ code style similar to default eclipse
\lstdefinestyle{eclipse-cpp} {
    captionpos=b,
    language=C++,
    otherkeywords={final},
    basicstyle=\footnotesize,
    numbers=left,
    numberstyle=\small,
    showstringspaces=false,
    tabsize=2,
    frame=single,
    breaklines=true,
    keywordstyle=\bfseries\color[RGB]{127,0,85},
    identifierstyle=\color[RGB]{0,0,192},
    stringstyle=\color[RGB]{42,0,255},
    commentstyle=\color[RGB]{63,127,95},
    aboveskip=1em,
    belowskip=1em,
}

\lstdefinestyle{standard} {
	basicstyle=\footnotesize\ttfamily, 
	captionpos=b,
	numbers=left,
	numberstyle=\small,
	showstringspaces=false,
	tabsize=2,
	frame=single,
	breaklines=true,
	aboveskip=1em,
	belowskip=1em,
}

% If no highlighting is intended
\lstdefinestyle{plain}{
	basicstyle=\ttfamily\footnotesize,
	aboveskip=1em,
	belowskip=1em,
}

%
\lstdefinelanguage{Kotlin}{
	comment=[l]{//},
	commentstyle={\color{gray}\ttfamily},
	emph={filter, first, firstOrNull, forEach, lazy, map, mapNotNull, println},
	emphstyle={\color{OrangeRed}},
	identifierstyle=\color{black},
	keywords={!in, !is, abstract, actual, annotation, as, as?, break, by, catch, class, companion, const, constructor, continue, crossinline, data, delegate, do, dynamic, else, enum, expect, external, false, field, file, final, finally, for, fun, get, if, import, in, infix, init, inline, inner, interface, internal, is, lateinit, noinline, null, object, open, operator, out, override, package, param, private, property, protected, public, receiveris, reified, return, return@, sealed, set, setparam, super, suspend, tailrec, this, throw, true, try, typealias, typeof, val, var, vararg, when, where, while},
	keywordstyle={\color{NavyBlue}\bfseries},
	morecomment=[s]{/*}{*/},
	morestring=[b]",
	morestring=[s]{"""*}{*"""},
	ndkeywords={@Deprecated, @JvmField, @JvmName, @JvmOverloads, @JvmStatic, @JvmSynthetic, Array, Byte, Double, Float, Int, Integer, Iterable, Long, Runnable, Short, String},
	ndkeywordstyle={\color{BurntOrange}\bfseries},
	sensitive=true,
	stringstyle={\color{ForestGreen}\ttfamily},
}

% fancy algorithms (see template)
\usepackage[ruled, vlined, linesnumbered]{algorithm2e}
\DontPrintSemicolon
\SetKw{KwBy}{by}
\SetKw{KwAnd}{and}

% clickable links and clickable table of content <3
% Options: links with linebreaks
\PassOptionsToPackage{hyphens}{url}
\usepackage[bookmarks=false]{hyperref}
\hypersetup{
    colorlinks,
    citecolor=black,
    filecolor=black,
    linkcolor=black,
    urlcolor=black
}
% Alterations to labels used by \autoref{}: Capitalize everyything
\def\chapterautorefname{Chapter}
\def\sectionautorefname{Section}
\def\subsectionautorefname{Subsection}
\def\algorithmautorefname{Algorithm}
\def\subfigureautorefname{Figure}
% for custon stuff like use:
% \hyperref[custom:foo]{Custom~\ref*{custom:foo}}

% -------------------------------------------------------------------------------
% ------------------------- Bibliography Customisation ------------------------
% -------------------------------------------------------------------------------

% replace \makebibliography with this package to enable nice formatting for citing web pages
\usepackage[%
backend=bibtex				% biber or bibtex
,style=numeric-comp		% numerical-compressed TODO: Check if plain is fine as style (was 'alpha' before I changed it)
,sortcites=true				  % sorts citations if multiple entry keys are passed to a citation command
,isbn=true
,url=true
,doi=true
%,natbib=true         % if you need natbib functions
]{biblatex}
\addbibresource{literature.bib}  % better than \bibliography
\usepackage{lipsum} % for filling pages with stuff
\usepackage[nohyperlinks]{acronym} % for all the abbreviations we will use. TODO: Check with option "printonlyused" if all acronyms appear in the document.

% -------------------------------------------------------------------------------
% --------------------------------- Thesis Info ---------------------------------
% -------------------------------------------------------------------------------

% set title, authors and stuff for the cover
% docytype needs xspace because it is used within text.
\def\doctype{Bachelor's Thesis\xspace}

\def\studyProgram{Informatics}
\def\title{Implementing a mobile app for object detection}

\def\titleGer{Entwicklung einer mobilen App zur Objekterkennung}
\def\author{David Drews}
% Prof
\def\supervisor{Univ.-Prof. Dr. Hans-Joachim Bungartz}
% PhD Candidate
\def\advisor{Severin Reiz, M.Sc.}
\def\date{15th of August 2021}

\begin{document}
\frontmatter
% -------------------------------------------------------------------------------
% ---------------------------------- COVERPAGE ------------------------------
% -------------------------------------------------------------------------------

% correct BCOR - undo at the end !!!
\def\bcorcor{0.15cm}
\addtolength{\hoffset}{\bcorcor}
\thispagestyle{empty}
\vspace{4cm}
\begin{center}
    \includegraphics[width=4cm]{templateStuff/tumlogo.pdf}\\[5mm]
    \huge DEPARTMENT OF INFORMATICS\\[5mm]
    \large TECHNICAL UNIVERSITY OF MUNICH\\[24mm]

    {\Large \doctype in \studyProgram}\\[20mm]
    {\huge\bf \title\par}
    \vspace{15mm}
    {\LARGE  \author}
    \vspace{10mm}
    \begin{figure}[h!]
        \centering
        \includegraphics[width=4cm]{templateStuff/informat.pdf}
   \end{figure}
\end{center}

\cleardoubleemptypage

% -------------------------------------------------------------------------------
% ---------------------------------- TITLEPAGE --------------------------------
% -------------------------------------------------------------------------------

\def\bcorcor{0.15cm}
\addtolength{\hoffset}{\bcorcor}
\thispagestyle{empty}
\vspace{10mm}
\begin{center}
    \includegraphics[width=4cm]{templateStuff/tumlogo.pdf}\\[5mm]
	\huge DEPARTMENT OF INFORMATICS\\[5mm]
	\large TECHNICAL UNIVERSITY OF MUNICH\\[24mm]
	{\Large \doctype in \studyProgram}\\[20mm]
	{\LARGE\bf \title}\\[10mm]
	{\LARGE\bf \titleGer}\\[10mm]
	\begin{tabular}{ll}
		\Large Author:      	& \Large \author \\[2mm]
		\Large Supervisor:  	& \Large \supervisor\\[2mm]
		\Large Advisor:			& \Large \advisor\\[2mm]
		\Large Submission Date:       		& \Large \date
	\end{tabular}
	\vspace{-1mm}
	\begin{figure}[h!]
		\centering
		\includegraphics[width=4cm]{templateStuff/informat.pdf}
	\end{figure}
\end{center}

% undo BCOR correction
\addtolength{\hoffset}{\bcorcor}
\newpage

% -------------------------------------------------------------------------------
% ---------------------------------- DISCLAIMER -------------------------------
% -------------------------------------------------------------------------------

\cleardoubleemptypage

\thispagestyle{empty}
\vspace*{0.7\textheight}
\noindent
I confirm that this \MakeLowercase{\doctype} is my own work and I have documented all sources and material used.\\

\vspace{15mm}
\noindent
Munich, \date \hspace{5cm} \author
\cleardoubleemptypage

% -------------------------------------------------------------------------------
% ---------------------------------- ABSTRACT --------------------------------
% -------------------------------------------------------------------------------

\phantomsection
\addcontentsline{toc}{chapter}{Abstract}
\vspace*{2cm}
\begin{center}
    {\Large \bf Abstract}
\end{center}
\vspace{1cm}

\lipsum[2]

\cleardoublepage

% -------------------------------------------------------------------------------
% ---------------------------------- ACRONYMS --------------------------------
% -------------------------------------------------------------------------------

\phantomsection
\addcontentsline{toc}{chapter}{Acronyms}
\vspace*{2cm}
\begin{center}
	{\Large \bf Acronyms}
\end{center}
\vspace{1cm}

\begin{acronym}[ResNet] % use the longest abbrevation as option to attain an uniform layout
	\acro{ml}[ML]{Machine Learning}
	\acro{cv}[CV]{Computer Vision}
	\acro{nn}[NN]{Neural Network}
	\acro{cnn}[CNN]{Convolutional Neural Network}
	\acro{resnet}[ResNet]{Residual Neural Network}
	\acro{relu}[ReLU]{Rectified Linear Unit}
\end{acronym}

\cleardoublepage

% -------------------------------------------------------------------------------
% ------------------------------ TABLE OF CONTENTS -------------------------
% -------------------------------------------------------------------------------

\tableofcontents
\thispagestyle{empty}
\cleardoubleemptypage

% -------------------------------------------------------------------------------
% --------------------------------- MAIN MATTER ------------------------------
% -------------------------------------------------------------------------------

\mainmatter
\chapter{Introduction}

\section{Growing Support for Running Machine Learning Operations on Mobile Devices}

\section{Privacy Implications of On-Device Machine Learning}

\chapter{Background Theory in Computer Vision}

\section{History of Computer Vision}

\section{Typical Tasks in the Field of Computer Vision}

\section{Computer Vision on Mobile Devices}

\chapter{State of the Art Solutions for Object Detection on Mobile Devices}

\section{Introduction to XYZ Networks}

\section{Some Deep}
\section{Dive Into}
\section{Object Detection}
\section{Theory Fun}

\chapter{App Development}

\section{Previous State of the Application}

\subsection{Use Cases}

\subsection{Notable Design Decisions}

\section{Development Goals}

\subsection{Migration From Java to Kotlin}

\subsection{New Functionality: Object Detection}

\section{Implementing Object Detection Based on the TensorFlow Lite Framework}

\subsection{Some Deep}
\subsection{Dive Into}
\subsection{Object Detection}
\subsection{Implementation Fun}

% Potential Contets: Design and architecture choices, reused patterns from exisiting application, improvements to exisiting code

\chapter{Results}

\section{Performance of Object Detection Algorithms on Mobile Devices}

\section{Future Work} % Can also be named "Outlook" depending on the contents of this section

% -------------------------------------------------------------------------------
% ----------------------------------- APPENDIX --------------------------------
% -------------------------------------------------------------------------------

\appendix

\chapter{Screenshots of the Application}

\chapter{Tips With Greetings From the Chair}
\label{sec:tips}       % labels can be put almost anywhere and can be referencef from anywhere.
Here are tips along the way:

\section{Tips}
\subsection{How to Describe}
% optional: set the spacing between columns
\setlength{\columnsep}{30 pt}
When listing several points you have three basic options:
\begin{multicols}{3}
	\begin{itemize}
		\item itemize
		\item enumerate
		\item description
	\end{itemize}
	
	\vfill\null
	\columnbreak
	
	\begin{enumerate}
		\item itemize
		\item enumerate
		\item description
	\end{enumerate}
	
	\vfill\null
	\columnbreak
	
	\begin{description}
		\item[itemize] short, unordered
		\item[enumerate] short ordered
		\item[description] listing of descriptions. Also nice for longer ones.
	\end{description}
	
\end{multicols}


\subsection{How to Quote}

\begin{quote}
	"This is a quote!"
\end{quote}

\begin{itemize}
	\item Citations to a source can be made like this \verb|\cite{gratl17task}| =~\cite{gratl17task}
	\subitem Always join text and the citation with a non-breaking space: \verb|text~\cite{foo}|.
	\item Referencing Sections, Figures, Tables, Formulas: \verb|\autoref{sec:tips}| = \autoref{sec:tips}.
	\item Footnotes for url or further notes: \verb|\footnote{\url{https://www.top500.org}}| = \footnote{\url{https://www.top500.org}}
\end{itemize}

\subsection{How to Math}

Use the align environment for equations especially if you want to align them somehow.

\begin{align}
	1 + 1 &\ne 3\\
	\left(\dfrac{10}{1}\right) - 9 &= 1
\end{align}

% if you need a pagebreak because figure placement is broken:
\clearpage

\section{Environments}

\subsection{How to Figure}

Anything can also be put in multiple columns.

\begin{multicols}{2} % defines an environment with two columns
	\begin{figure}[H] % [H] for HERE
		\centering
		\includegraphics[width=.9\columnwidth]{figures/scenario_clip_rot.png}
		\caption[Example Figure]{Some Caption. Always also include a source if it wasn't created by you!\\
			\tiny{Source: \cite{gratl17task}}}
		\label{fig:exampleLabel1} % labels always have to be placed after the caption
	\end{figure}
	
	\columnbreak    % start next column
	
	\begin{figure}[H]
		\centering
		\begin{tikzpicture}
			\node[anchor=south west,inner sep=0] (image) at (0,0) {\includegraphics[width=.9\columnwidth]{figures/scenario_clip_rot.png}};
			\begin{scope}[x={(image.south east)},y={(image.north west)}]
				\draw[red, thin,rounded corners] (.42,.42) rectangle (.58,.6);
			\end{scope}
		\end{tikzpicture}
		\caption[Figure with tikz]{Figures can be drawn on or completely generated with tikz.}
		\label{fig:exampleLabel2}
	\end{figure}
\end{multicols}

\paragraph{Subfigures}
If grouping of several pictures seems reasonable, think about using subfigures. This often comes in handy with plots.

\begin{figure}[H]
	\centering
	\begin{subfigure}[b]{0.33\textwidth}
		\includegraphics[width=\textwidth]{example-image-a}
		\caption{example-image-a}
		\label{fig:example-image-a}
	\end{subfigure}
	\begin{subfigure}[b]{0.33\textwidth}
		\includegraphics[width=\textwidth]{example-image-b}
		\caption{example-image-b}
		\label{fig:example-image-b}
	\end{subfigure}
	\begin{subfigure}[b]{0.33\textwidth}
		\includegraphics[width=\textwidth]{example-image-c}
		\caption{example-image-c}
		\label{fig:example-image-c}
	\end{subfigure}
	\caption{One caption to describe them all.}
\end{figure}

\subsection{How to Algorithm}

\begin{figure}
	\begin{algorithm}[H]
		
		% Define custom keywords
		\SetKwFunction{KwNot}{not}
		% Define custom Functions
		\SetKwFunction{Fissorted}{is\_sorted}
		\SetKwFunction{Fbogosort}{bogosort}
		\SetKwFunction{Fshuffle}{shuffle}
		\SetKwProg{Fn}{Function}{:}{}
		\KwIn{\tabto{2cm}data array}
		\KwOut{\tabto{2cm} data sorted}
		\BlankLine
		
		\tcp{Checks if array is sorted}
		\Fn{\Fissorted{data}}{
			\For{i $\leftarrow$ 0 \KwTo data.size() - 1}{
				\label{algo:for}            % labels can also be put in the algorithm
				\If{data[i] $>$ data[i+1]}{
					\Return false
				}
			}
			\Return true
		}
		
		\tcp{actual algorithm}
		\Fn{\Fbogosort{data}}{
			\While{\KwNot \Fissorted{data}}{
				random.\Fshuffle{data}
			}
		}
		
		\caption[Bogosort]{Bogosort}
		\label{algo:example}
	\end{algorithm}
	\caption{some description what is happening}
\end{figure}

\clearpage

\subsection{How to Code}
\begin{lstlisting}[style=eclipse-cpp, caption=General form of a typical runner() function., label=code:runner]
	void runner(int type, void *data){
		switch(type)
		case taskType1:
		// do stuff using data
		case taskType2:
		// do other stuff using data
	}
\end{lstlisting}

\subsection{How to Table}
\begin{table}[H]
	\begin{tabularx}{\columnwidth}{L | C | R}
		\hline
		\hline
		bla left & bla centered\newline over two lines &  bla right\\
		\hline
		bla left & bla centered & \multirow[c]{2}{\hsize}{cell spanning two rows} \\
		\cline{1-2}
		\multicolumn{2}{c|}{cell spanning two columns} & \\
	\end{tabularx}
	\caption[Some Table]{Fancy table that can contain line breaks and extended cells.}
	\label{tab:example}
\end{table}

%TODO: Insert screenshots. Potentially even with the respective old version of a screen to see the development.

\listoffigures

\listoftables

\bibliographystyle{alpha}
\bibliography{literature}

\end{document}
